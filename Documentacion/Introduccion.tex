
%%%%%%%%%%%%%%%%%%%%%%%%%%%%%%%%%%%%%%%%%%%%%%%%%%%%%%%%%%%%%%%%%%%
%%% Documento LaTeX 											%%%
%%%%%%%%%%%%%%%%%%%%%%%%%%%%%%%%%%%%%%%%%%%%%%%%%%%%%%%%%%%%%%%%%%%
% Título:	Plantilla de TFG/TFM
% Autor:  Carlos Andres Goez, Juan Pablo Baena 
% Fecha:  2020-03-12
%%%%%%%%%%%%%%%%%%%%%%%%%%%%%%%%%%%%%%%%%%%%%%%%%%%%%%%%%%%%%%%%%%%
%	Modo:					PDFLaTeX.
%%%%%%%%%%%%%%%%%%%%%%%%%%%%%%%%%%%%%%%%%%%%%%%%%%%%%%%%%%%%%%%%%%%
%------------------------------

\Large 
\textbf{Introducción}

Actualmente en Colombia hay 8004 hectáreas de flores sembradas, donde 1500 hectáreas corresponden al género Hydrangea también conocidas como hortensias, las cuales el 99\% de estas se encuentran en el oriente en donde se encuentran registrados 1381 predios productores de este tipo de flor.

La producción de hortensias en el Oriente Antioqueño se caracteriza por hacer un uso intensivo de mano de obra, aplicando poca o nula automatización en sus procesos, en donde los trabajadores de campo de los predios floricultores sufren desgates físicos por las condiciones de trabajo poco ergonómicas y repetitivas que tienen en su jornada laboral. La automatización podría ayudar a reducir este desgaste de los trabajadores implementándola en algunos procesos, además, podría traer otros beneficios como reducir el costo de mano de obra, mejorar la calidad en el producto, reducir el tiempo de manufactura, entre otros. Sin embargo, el bajo poder adquisitivo, la flexibilidad que se debe tener en la producción de este tipo de flor y el componente social relacionado con el reemplazo de la mano de obra debido a la automatización se presentan como una barrera a su implementación, por lo que sistemas semiautomáticos pueden ser una opción más viable. 

El Oriente antioqueño posee una topografía en su mayoría montañosa, por lo que implementar la automatización en áreas como la siembra y la cosecha presentan mayor grado de dificultad; además, no se encuentran estandarizados estos procesos en todas las floricultoras, sin embargo, post cosecha cuenta con procesos más unificados y estándares en los cultivos de hortensias del Oriente Antioqueño.

En el presente trabajo se prototipa un sistema semiautomático post cosecha de Hortensias que cumpla con las funciones de clasificación, empaque individual en capuchón y monitoreo. 