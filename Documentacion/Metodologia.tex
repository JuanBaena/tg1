
%%%%%%%%%%%%%%%%%%%%%%%%%%%%%%%%%%%%%%%%%%%%%%%%%%%%%%%%%%%%%%%%%%%
%%% Documento LaTeX 											%%%
%%%%%%%%%%%%%%%%%%%%%%%%%%%%%%%%%%%%%%%%%%%%%%%%%%%%%%%%%%%%%%%%%%%
% Título:	Plantilla de TFG/TFM
% Autor:  Carlos Andres Goez, Juan Pablo Baena 
% Fecha:  2020-03-12
%%%%%%%%%%%%%%%%%%%%%%%%%%%%%%%%%%%%%%%%%%%%%%%%%%%%%%%%%%%%%%%%%%%
%	Modo:					PDFLaTeX.
%%%%%%%%%%%%%%%%%%%%%%%%%%%%%%%%%%%%%%%%%%%%%%%%%%%%%%%%%%%%%%%%%%%
%------------------------------

\chapterbegin{Metodología}

\minitoc

Para el desarrollo del proyecto se utiliza la metodología de Karl Ulrich y Steven Eppingger  expuesta en su libro "Diseño y desarrollo de productos", esta metodología se divide en cinco etapas las cuales van desde la planeación y primeras ideas de los productos hasta producción.
\section{Planeación}
Se plantean las ideas iniciales sobre el proyecto, se plantean los objetivos a alcanzar con el proyecto, además se declaran los supuestos y restricciones que tendrá el proyecto.
\section{Desarrollo del concepto}
Se realiza una recolección de datos en diferentes cultivos de hortensias con el fin, conocer los procesos más críticos y más viables a automatizar dentro del cultivo de hortensias,
además se definen las especificaciones generales del producto.A partir las especificaciones se plantean, subsistemas para cada proceso a automatizar. Previo a esto se realiza una generación de conceptos a partir de las funcionalidades que debe tener cada subsistema y se selecciona el concepto que mejor cumpla con los requerimientos antes obtenidos.
\section{Diseño a nivel sistema}
A partir de los conceptos obtenidos se realiza una prueba de concepto mediante un diagrama de bloques donde se detallan los componentes a utilizar, con este diagrama se puede observar con facilidad las falencias del concepto evaluado. A partir de este diagrama de bloque se establece la arquitectura del proyecto y un diseño geométrico del producto.
\section{Diseño de detalle}
Teniendo la arquitectura del sistema, se realiza el diseño para manufactura de sistemas los sistemas electrónicos, mecánicos y software donde se especifican medidas, tolerancias y materiales. A partir de los diseños se realizaran simulaciones para verificar su funcionamiento, después de realizar estas pruebas se obtendrán los planos de ensamble del sistema y especificaciones técnicas.
\section{Pruebas y refinamiento}
A partir de los planos y especificaciones técnicas, se realiza el prototipado de la máquina donde se integran cada uno de los subsistemas. Teniendo este prototipo se plantea un esquema de pruebas donde se evalúan las especificaciones de la maquina y cumplimiento de los objetivos planteados.
%\section{Inicio de producción}



\chapterend{}
