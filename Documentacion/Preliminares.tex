
%%%%%%%%%%%%%%%%%%%%%%%%%%%%%%%%%%%%%%%%%%%%%%%%%%%%%%%%%%%%%%%%%%%
%%% Documento LaTeX 											%%%
%%%%%%%%%%%%%%%%%%%%%%%%%%%%%%%%%%%%%%%%%%%%%%%%%%%%%%%%%%%%%%%%%%%
% Título:	Plantilla de TFG/TFM
% Autor:  Carlos Andres Goez, Juan Pablo Baena 
% Fecha:  2020-03-12
%%%%%%%%%%%%%%%%%%%%%%%%%%%%%%%%%%%%%%%%%%%%%%%%%%%%%%%%%%%%%%%%%%%
%	Modo:					PDFLaTeX.
%%%%%%%%%%%%%%%%%%%%%%%%%%%%%%%%%%%%%%%%%%%%%%%%%%%%%%%%%%%%%%%%%%%
%------------------------------
% !TEX root =  Documento_TG.tex

\chapterbegin{Introducción y visión general}
\minitoc

\section{Planteamiento del problema}

\section{Objetivo}
\label{sec:intro:obj}
En esta sección, se describe el \miindex{objetivo del proyecto}, es decir, qué pretende, a qué aspira, cuál es su meta. Es importante comprender esta sección, porque de otro modo, no se entiende el resto de la documentación.

\section{Estado del arte}

Un proyecto se realiza sobre un \miindex{estado de la técnica} que debe explicarse para entender mejor conceptos tales como los problemas existentes o cuáles son las soluciones que se emplean hasta la fecha actual. El \miindex{estado del arte}, a veces llamado estado de la técnica, suele estar presente en este tipo de documentos.


\chapterend
